\section{The Ball In and Out of Play}\label{sec:ball-in-and-out-of-play}

\subsection{Ball Out of Play}
The ball is out of play when:
\begin{itemize}
\item it has wholly crossed the goal \added{line} \removed{boundary} or touch \added{line} \removed{boundary} whether on the ground or in the air
\item play has been stopped by a signal from the referee
\end{itemize}

\subsection{Ball Placement}
When the ball goes out of play, robots should remain 500\,mm from the ball as the ball is placed until the restart signal is given by the referee.
\removed{The referee or the assistant referee will either manually place the ball or indicate the position of ball placement to one of the teams.}
\added{
The automatic referee will indicate the position of ball placement to the team that will bring the ball into play:
}
\added{
\begin{itemize}
\item The human referee has to place the ball for all kickoffs.
\item For an indirect freekick, direct freekick or penalty kick, the team that brings the ball into play must place the ball.
\item For a forced start, a team is drawn by chance and must place the ball.
\item The ball must be visible and must not be inside a field or goal corner or behind the goal, before the ball placement starts.
\item The human referee can decide to place the ball manually, if ball placement is not feasible for a robot.
\item When a team has failed to place the ball five times in a row, it is not allowed to place the ball for the rest of the game half. All freekicks are awarded to the opposing team. If the team is awarded a penalty kick, the ball is placed by the human referee.
\item If no team can place the ball, the ball is placed by the human referee.
\end{itemize}
}

\added{
Ball placement is mandatory for all teams in Division A. Teams in Division B may decide, at any time before or during the game, not to place the ball for the rest of the game. In this case, the team is allowed to bring the ball into play, after the ball was placed by the opposing team. If the opposing team fails to place the ball or no team can place the ball, it is placed by the human referee.
}

If a team is assigned with a ball placement task, the play is considered as stopped
with the exception that the placing team is allowed to approach and touch the ball.
A ball is considered placed successfully if

\begin{itemize}
\item no more than \removed{15} \added{30} seconds passed since the placement command
\item there is no robot within 500\,mm distance to the ball
\item the ball is stationary
\item the ball is at a position within 100\,mm radius from the requested position
\end{itemize}

\added{
No further commands will be issued by the Referee Box until the automatic placement is complete. The game will be continued by the Referee Box as soon as the ball is successfully placed. A failed placement will result in an indirect free kick for the opposing team.
}

\removed{If a team repeatedly fails to place the ball, the referee may stop assigning
ball placement tasks to this team. }
The non-placing team should avoid the ball on a best-effort approach.



\subsection{Ball In Play}
The ball is in play at all other times.

\subsection{Infringements/Sanctions}
If, at the time the ball enters play, a member of the kicker's team occupies the area closer than 200\,mm to the opponent's defence area:
\begin{itemize}
\item an indirect free kick is awarded to the opposing team, the kick to be taken from the location of the ball when the infringement occurred (see \autoref{sec:free-kicks})
\end{itemize}

If, after the ball enters play other than due to a forced restart, the kicker touches the ball a second time (without holding the ball) before it has touched another robot:
\begin{itemize}
\item an indirect free kick is awarded to the opposing team, the kick to be taken from the place where the infringement occurred (see \autoref{sec:free-kicks})
\end{itemize}

If, after the ball enters play other than due to a forced restart, the kicker deliberately holds the ball before it has touched another robot:
\begin{itemize}
\item a direct free kick is awarded to the opposing team, the kick to be taken from the place where the infringement occurred (see \autoref{sec:free-kicks})
\end{itemize}

If, after a signal to restart play is given, the ball does not enter play within 10 seconds or lack of progress clearly indicates that the ball will not enter play within 10 seconds:
\begin{itemize}
\item the play is stopped by a signal from the referee,
\item all robots have to move 500\,mm from the ball
\item a forced restart is indicated, and
\item once the referee indicates the forced restart, robots from either team may approach and touch the ball
\end{itemize}

\subsection*{Decisions of the Small Size League Technical Committee}
\begin{enumerate}
\item
For all restarts where the Laws stipulate that the ball is in play when it is kicked and moves, the robot must clearly tap or kick the ball to make it move.
It is understood that the ball may remain in contact with the robot or be bumped by the robot multiple times over a short distance while the kick is being taken, but under no circumstances should the robot remain in contact or touch the ball after it has traveled 50\,mm, unless the ball has previously touched another robot.
Robots may use dribbling and kicking devices in taking the free kick.

\item
The exclusion zone closer than 200\,mm to the opponent's defence area during restarts is designed to allow defending teams to position a defence against a kick without interference from the opponents.
This change was added to help teams defend against corner kicks in which teams use elevated ``chip kick'' passes directly into the defence area.
\end{enumerate}
