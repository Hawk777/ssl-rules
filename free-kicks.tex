\section{Free Kicks}\label{sec:free-kicks}

\subsection{Types of Free Kicks}
Free kicks are either direct or indirect.

For both direct and indirect free kicks, the ball must be stationary when the kick is taken and the kicker does not touch the ball a second time until it has touched another robot.

\subsection{The Direct Free Kick}
\begin{itemize}
\item if a direct free kick is kicked directly into the opponents' goal, a goal is awarded to the
kicking team
\item if a direct free kick is kicked directly into the team's own goal, a corner kick is awarded
to the
opposing team
\end{itemize}

\subsection{The Indirect Free Kick}
A goal can be scored only if the ball subsequently touches another robot before it enters the goal.

\begin{itemize}
\item if an indirect free kick is kicked directly into the opponents' goal, a goal kick is awarded
to the opposing team.
\item if an indirect free kick is kicked directly into the team's own goal, a corner kick is
awarded to the opposing team.
\end{itemize}

\subsection{Free Kick Procedure}
If the free kick is awarded to a team inside or within 200\,mm of its own defence area, the free kick is taken from a point 600\,mm from the goal line and 100\,mm from the touch line closest to where the infringement occurred.

If the free kick is awarded to the attacking team within 700\,mm of the opposing defence area, the ball is moved to the closest point 700\,mm from the defence area.

Otherwise, the free kick is taken from the place where the infringement occurred.

All opponent robots are at least 500\,mm from the ball.

The ball is in play when it is kicked and moves.

\subsection{Infringements/Sanctions}
If, when a free kick is taken, an opponent is closer to the ball than the required distance:
\begin{itemize}
\item the kick is retaken
\end{itemize}

Any infringement as listed in \autoref{sec:ball-in-and-out-of-play} is handled accordingly.

For any other infringement of this Law:
\begin{itemize}
\item the kick is retaken
\end{itemize}
